\documentclass[a4paper,11pt]{report}
\usepackage{cite}
\usepackage{graphicx}
\usepackage{wrapfig}
\usepackage{appendix}
\usepackage{chngcntr}
\usepackage{caption}

\counterwithout{figure}{chapter}

%%%%  Add some length to the page, as margins always seem too big.  
\addtolength{\topmargin}{-.5in}

\begin{document}

%%%%  Number the initial front matter with roman numerals
\pagenumbering{roman}


%%%%  TITLE PAGE
%%  Here are the elements that make up the title page of the document.  

\thispagestyle{empty}

\title{\LARGE
ASDA Stocking Support System (Xpire)}

\author{
Luke Michael Towell
\\    \\    \\
Final Report
\\    \\
Submitted to 
\\    \\    \\ 
The University of Liverpool
\\    \\
\\    \\
in partial fulfilment of the requirements
\\
for the degree of 
\\     \\
MASTER OF SCIENCE
\\     \\    \\    \\
}


%%  Fill in a date here if you want, or comment out the line below and
%%  the current date will be automatically inserted for you.  
\date{}


\maketitle


%%%%  ABSTRACT
\chapter*{\center Abstract}

The aim of the project is to create an application which enables ASDA colleagues to identify
and inform colleagues of which items within their fresh departments are going out of date and
need to be marked down or could potentially be provided to food shelters. This project will make use of mobile technology,
 backend web services and Database stored procedures in order to analyse and inform colleagues which items are Scheduled to go 
out of date on which date and then informs colleagues to go and reduce the products. This application
will be used throughout ASDA stores by colleagues on a daily basis and could produce a significant
cost saving and waste reduction.
\\
\\
The final products of this project will be a system of applications which will be ready for deployment
into an ASDA store in the future. The requirements and periodic demoing of the produced solution will
be provided by ASDA technology colleagues.

\newpage

%%%%  STUDENT DECLARATION ON PLAGIARISM
\chapter*{\center Student Declaration} 

I confirm that I have read and understood the University's Academic Integrity Policy.

I confirm that I have acted honestly, ethically and professionally in conduct leading
to assessment for the programme of study.  

I confirm that I have not copied material from another source nor committed plagiarism
nor fabricated data when completing the attached piece of work.  I confirm that I have 
not previously presented the work or part thereof for assessment for another University
of Liverpool module.  I confirm that I have not copied material from another source, nor
colluded with any other student in the preparation and production of this work.  

I confirm that I have not incorporated into this assignment material that has been 
submitted by me or any other person in support of a successful application for a 
degree of this or any other university or degree-awarding body.  

\vspace*{1in}

\noindent SIGNATURE \verb!______________________________________!

\noindent DATE \hspace*{.4in}  \today

\vspace*{1in}

%%  NOTE ABOUT CONFIDENTIAL MATERIAL  
%%     Students who need to keep their dissertation confidential should uncomment
%%     the following sentence on this same page.  This will preclude the dissertation 
%%     from being placed in the University Library.  Students that submit work that 
%%     isn't confidential should leave this line commented out.  

% This dissertation contains material that is confidential and/or commercially
% sensitive. It is included here on the understanding that this will not be revealed to 
% any person not involved in the assessment process.  


\newpage

%%%%  ACKNOWLEDGMENTS
%%%%    A section for Acknowledgments, should you want one.



\newpage


%%%%  TABLE OF CONTENTS  
%%%%      Usually the following command will give you the formatting you want.  

\tableofcontents


%%%%  Turn page numbering back to arabic.  This also resets the numbering
%%%%  to begin again at page 1.  

\pagenumbering{arabic}

%%%%  INTRODUCTION

\chapter{Project Introduction}\label{chap:intro}

\section{Summary of Project Proposal}

\subsection{Problem Statement}\label{sec:problem}
The aim of this project is to design an application and waste management system which is able
to inform store managers and store colleagues on which items of stock are going out of date or
close to going out of date on the shop floor. This stock will then either be marked 
down on the shop floor in order to be sold on the day or will be identified as able to be donated to other charitable
causes within the local community as part of the ASDA commitment to engage and assist in the local community.
\\
\\
The key goals of the developed system are to include:
\\
- An application which will direct colleagues to which items need to be marked down.
\\
- The hours spent in store manually marking down products will be reduced.
\\
- Stock which is wasted will be reduced and identified for redistribution.
\\
\\
These goals will be assessed by peer review of colleagues in ASDA. I am also aiming to 
deploy the application in a live ASDA store with the aim of it being used in a live environment.
If I am able to deploy a POC into a live store then I will also assess the success of the 
application by comparing the product waste prior to usage in store against the wastage from 
the time that the application is used in store. 

\subsection{Project Methodology}
The methodology that I have chose to implement for my project was a scrum agile methodology
which involved breaking the project down into smaller requirements which needed to be completed in 
iterations of development spread over the 8 weeks I developed my project in.  
\\
\\
The project included the management and development of multiple application features. Each of the features were 
detailed and documented on a Trello kanban board (See Figure~\ref{fig:kanbanBoard}).  As can be seen in Figure 1 the kanban board is broken down into "To Do", "Doing" and "Done" swim lanes 
which allows the user to easily identify the state of the feature that is being developed. 
In figure 2 you can see the updated Kanban board as I have been working on the project and moving the different requirements across the board as they have progressed.
% TODO: Add in figure of Kanban board

\chapter{Final Outcomes}
The aim of this project from the outset was to create a software system which makes use of multiple applications
 to give the in store user a clean and easy to use way of managing in store waste. In the following chapter I 
 discuss what has been produced and I discuss some of the user flows which take place within the system.
\section{Output of the project}
% TODO: CITE SPECIFICATION AND DESIGN DOCUMENT
\subsection{User Interface Design}
As part of the development of the system a key to ensuring the ease of use of the application was to take care and attention when
designing the user interface of the different applciations. In order to do this I made use of Balsamic in order to design the user
interface of both the mobile and web applications. Appendix 2 shows the outcomes of the designs, Image 4 shows an example of the 
interface design in Balsamiq and the final user interface that was produced.
\subsection{Database Schema Design}
\subsection{Developed Applications}
\subsubsection{Mobile Application}
\subsubsection{Web Application}
\subsubsection{Web Services}

\section{Project Review}
\subsection {Successes of the project}

\subsection {Challenges of the Project}
Throughout the project there have been issues which have arisen an the solution has been developed. In the early stages of the project it was identified that due to data security and authentication security utilised by ASDA I would be unable to make use of data and Authentication systems. This has presented various different issues which include:
\begin{itemize}
    \item Data availability
    \item User Authentication
    \item Application and Database hosting
    \item Store Usage
\end{itemize}

All of these issues have been over come or worked around in order to produce a working software system. Each of the Issues listed above will be discussed in more detail along with the relevant work arounds in the dissertation of this project.
\section{Solution Evaluation}

\subsection{Colleague Feedback}

\chapter{Potential Expansion}
\section{Potential Project Expansions and Features}
The following chapter includes the potential expansions which could be applied to the application after it has been completed.
 These would require more research and could potentially be projects in their own right.

\subsection{Timesheeting}
Through the use of data mining from the data collected by the application, we should be able to accurately predict
how long it takes staff to organise and mark down the expiring items within their respective departments. Once 
this data has been collected then theoretically it could be used in order to calculate how many staff are needed 
in order to mark down items on a particular day depending on the number of items that need to be reduced as well 
as the length of time that it is likely to take the staff in order to mark down or redistribute the expiring stock
on the shop floor. 

\subsection{Stock Prediction and Inventory Management}

By analysing the number of items which are expiring in individual stores we should also be able to perform trend 
analysis on the data which should highlight items of stock which are repeatedly marked down. This trend analysis 
could then be used to inform store managers and the ASDA inventory management systems of any over ordering which is 
occuring and allow for more accurate stock levels to reduce potential waste in the future.
%%%%   REFERENCES

%%%%  Section for references, using the \bibitem directive to 
%%%%  specify labels used to cite sources in the document.  
\newpage
\addcontentsline{toc}{chapter}{Appendices}
\begin{appendix}
\chapter{}\label{app:ganntchart}
\begin{figure}[ht]
    \centering
    \includegraphics[scale=1.0]{./assets/images/gannt1.png}
    {\caption*{Xpire Project Gannt Chart June - July}}
\end{figure}
\begin{figure}[ht]
    \centering
    \includegraphics[scale=1.0]{./assets/images/gannt2.png}
    {\caption*{Xpire Project Gannt Chart August - September}}
\end{figure}
\end{appendix}

\bibliography{bibliography}
\bibliographystyle{plain}

%%%%   APPENDICES


\end{document}
